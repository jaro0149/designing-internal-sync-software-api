Modeling of software API (Application Programming Interface) is essential part of the software design
and development process.
Making bad decisions during the design phase can lead to a lot of issues in the future including
increased maintenance cost, decreased flexibility of API, and stopping next API evolution.
These issues are also multiplied by the fact that API is used by other components in the system.

This article focuses on the most commonly used design patterns and approaches that can be leveraged during creation
or refactoring of internal synchronous software API. Every approach has its own benefits and drawbacks and should be
used in the right context.
By internal API we mean API that is used by other components in the same process between different software components
such as modules or classes, usually described by some interfaces.
However, some of the described patterns can be also used for modeling of external or asynchronous interfaces.

The article focuses more on the structure of the API than how components are communicating with each other.
Also, the article does not focus on the implementation details of the API such as specific technologies or languages,
but rather on the general design principles.

The paper is structured as follows.
The first part describes the requirements and analysis of target domain that will be used as the example.
Next sections describe design approaches and related design patterns demonstrated on the example using UML diagrams
starting from the Singleton Interface Method approach, through object-oriented approaches,
and ending with the model-driven API\@.
The last part summarizes the content using metamodel view on all the approaches.
