Modeling a software API (Application Programming Interface) is an essential part of the software design
and development process.
Making poor decisions during the design phase can lead to numerous issues in the future, including increased
maintenance costs, decreased flexibility of the API, and hindering the evolution of the next API\@.
These issues are further compounded by the fact that the API is utilized by other components in the system.

This guide focuses on the most commonly used design patterns and approaches that can be leveraged during
the creation or refactoring of internal synchronous software APIs. Each approach has its own benefits
and drawbacks and should be employed in the right context.
When we refer to an internal API, we mean an API that is used by other components in the same process between
different software components, such as modules or classes, typically described by specific interfaces.
However, some of the described patterns can also be applied to model external or asynchronous interfaces.

The guide emphasizes the structure of the API more than how components communicate with each other.
Additionally, the document does not delve into the implementation details of the API,
such as specific technologies or languages, but rather focuses on general design principles.

The paper is structured as follows.
The first part describes the requirements and analysis of the target domain, which will be used as an example.
The subsequent sections detail design approaches and related design patterns,
demonstrated on the example using UML diagrams.
This starts with the Singleton Interface Method approach, progresses through object-oriented approaches,
and concludes with the Model-Driven API\@.
The final part provides a summary of the content, presenting a metamodel view of all the approaches.
